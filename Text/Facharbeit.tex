\documentclass[12pt]{article}

\usepackage{german}
\usepackage{geometry}
\usepackage[onehalfspacing]{setspace}

\geometry{
    left=3cm,
    right=3cm,
    top=2.5cm,
    bottom=2.5cm
}

\title{Numerisches finden von Nullstellen - Implementierung in Java}
\author{Jakob Rzeppa}

\begin{document}
\begin{titlepage}
	\centering
    {\huge IGSFF LOGO TODO\par}
	{Grünewaldstraße 12a - 38104 Braunschweig - T. 470 5850\par}
	\vspace{1cm}
	{\underline{Seminarfach:}\par}
	\vspace{1cm}
    {\large Facharbeit des Schülers \textit{Jakob Rzeppa} \par mit dem Thema: \par}
    \vspace{1.5cm}
	{\huge Numerisches finden von Nullstellen - Implementierung in Java\par}
	\vspace{2cm}
\end{titlepage}

\tableofcontents

\section{Einleitung}
    \subsection{Problemstellung}

\section{Newton-Horner-Methode}
    \subsection{Herleitung und Funktionsweise}
    \subsubsection{Verfahren}
    \subsubsection{Endbedingungen}
    \subsubsection{Startpunkte}
    \subsection{Beweis}
    \subsection{Probleme}
    \subsection{Implementierung}

\section{Durand-Kerner-Methode} 
    Die Weierstraß-(Durand-Kerner)-Methode ist ein iteratives Verfahren, zur numerischen Bestimmung aller Nullstellen einer Polynomfunktion. Sie wurde zwischen 1859 und 1891 von Karl Weierstrass, als Teil seines Beweises zum Fundamentalsatz der Algebra, entwickelt. E. Durand und Immo Kerner überführten dieses Verfahren in einen Computeralgorythums.

    \subsection{Herleitung und Funktionsweise}
    \subsubsection{Verfahren}
        Sei $f(x) = x^n + a_{n-1} x^{n-1} + ... + a_1 x + a_0$ ein univariates Polynom mit komplexen Koeffizienten und führenden Koeffizienten $a_n = 1$.
        Nach dem Fundermentalsatz der Algebra hat $f(x)$ genau $n$ Nullstellen $\xi_1;...;\xi_n \in \mathbb{C}$ und kann in Linearfaktoren zerlegt werden.
    \subsubsection{Endbedingungen}
    \subsubsection{Startpunkte}
    \subsection{Beweis}
    \subsection{Probleme}
    \subsection{Implementierung}

\section{Vergleich und Anwendung}

\section{Fazit}

\section{Code}

\end{document}