\documentclass[12pt]{article}

\usepackage{german}
\usepackage{geometry}
\usepackage[onehalfspacing]{setspace}

\geometry{
    left=3cm,
    right=3cm,
    top=2.5cm,
    bottom=2.5cm
}

\usepackage{graphicx}
\usepackage{float}

\usepackage{listings}
\usepackage{xcolor}

\usepackage{amsmath}
\usepackage{amssymb}

\usepackage{url}

\usepackage{listings}
\lstset{language=Java,
	basicstyle=\small,
	keywordstyle=\color{blue!80!black!100},
	identifierstyle=,
	commentstyle=\color{green!50!black!100},
	stringstyle=\ttfamily,
	breaklines=true,
	numbers=left,
	numberstyle=\small
}

\title{Numerisches finden von Nullstellen - Implementierung in Java}
\author{Jakob Rzeppa}

\begin{document}

\tableofcontents

\section{Einleitung}
Mithilfe von Nullstellen kann viel über den Verlauf einer Polynomfunktion herausgefunden werden. Polynome ersten bis vierten Grades lassen sich mithilfe von Formeln wie der pq-Formel exakt berechnen. Für Polynome des Grades $n > 4$ gibt es keine allgemeine Formel zur bestimmung der Nullstellen. Für diese muss auf einen anderen Teilbereich der Mathematik zurückgegriffen werden: die Numerik. In der Numerik versucht man durch Algorithmen und Annäherungen Nullstellen möglichst genau zu bestimmen. Händisch kann dies sehr lange dauern, jedoch kann mit Computern dieser Prozess extrem beschleunigt werden. Doch wie kann man Nullstellen von Polynomen mithilfe des Computers approximieren?\\
\\
Ziel dieser Facharbeit ist es, eine Möglichkeit der Implementierung eines numerischen Verfahrens zu entwickeln und dieses darzustellen. Dafür wird zuerst ein Verfahren ausgesucht und ausführlich beschrieben. Dabei werden weiterführend die Herleitung, Probleme und Einschränkungen behandelt und das Verfahren wird anhand eines Beispiels verdeutlicht. Im letzten Teil der Facharbeit wird meine Implementierung beschrieben und erklärt.\\
Zu dem Thema der Facharbeit bin ich gekommen, da ich mich seid der Einführung von Polynomfunktionen für das Bestimmen der Nullstellen von Polynomfunktionen höheren Grades interessiere. %TODO weiterschreiben
\\
Um ein Verfahren auszuwählen, dieses zu erklären und zu implementieren, werde ich neben Fachliteratur und Informationen von unterschiedlich Universitäten einen Teil der Informationen aus dem Internet ziehen, da nur wenig zu diesem spezifischen Thema zu finden ist. Dabei werde ich die Richtigkeit dieser Informationen möglichst genau überprüfen. %TODO ist das gut?

\section{Auswahl des Verfahren}
\subsection{Vorstellung der Verfahren}
In der Numerik gibt es viele verschiedene Verfahren für die Nullstellenberechnung. Die bekanntesten Verfahren sind die Newton-Methode\footnote{
    Prof. Dr. Bernd Engelmann: Numerische Mathematik, 2.4 Das Newton-Verfahren und seine Abkömmlinge (S.24)
} und das Bisektionsverfahren bzw. der Zwischenwertsatz\footnote{
    mathe-online.at: Zwischenwertsatz und Bisektionsverfahren, URL: \url{https://www.mathe-online.at/nml/materialien/innsbruck/bisektion/Bisektion.pdf}
}. Beide Methoden sind jedoch nur für das Finden einer Nullstelle geeignet. Da das Ziel ist alle Nullstellen eines Polynoms zu finden, kommen sie nicht in Frage.\\
Wenn man alle Nullstellen finden möchte, muss man komplexere Verfahren in Betracht ziehen. Dabei bieten sich Weiterentwicklungen der Newton-Methode, wie die Weierstraß-Iteration (Durand-Kerner-Methode)\footnote{
    vgl. Wikipedia: Durand-Kerner method, URL: \url{https://en.wikipedia.org/wiki/Durand-Kerner_method} (Zuletzt aufgerufen: 29.10.2023)
    \label{ftn:Wikipedia-Durand-Kerner-Method}
}, das Aberth-Ehrlich-Verfahren\footnote{
    Wikipedia: Aberth method, URL: \url{https://en.wikipedia.org/wiki/Aberth_method} (Zuletzt aufgerufen: 18.11.2023)
} oder das Newton-Horner-Verfahren\footnote{
    Eine Kombination aus der Newton-Methode und dem Horner-Verfahren.
} an.

\subsection{Vergleich der möglichen Verfahren}
Wichtige Kriterien bei dem vergleich von numerischen Verfahren sind die Geschwindigkeit und Einfachheit der Implementierung.\\
Die Weierstraß-Iteration und das Aberth-Ehrlich-Verfahren finden Nullstellen simultan, während das Newton-Horner-Verfahren die Nullstellen nacheinander findet. Aus diesem Grunde kommen sie mit weniger Iterationen aus. Das muss nicht immer bedeuten, dass diese Verfahren schneller sind. Jedoch können Prozesse parallel ausgeführt werden, was zu großen Verbesserungen der Performance führen kann.\\
Das Aberth-Ehrlich-Verfahren ist deutlich schneller als die Weierstraß-Iteration, benutzt jedoch eine komplexere Gleichung und ist deshalb schwieriger zu implementieren.\\
Das Newton-Horner-Verfahren ist schwierig zu implementieren, da es eine sehr hohe bis exakte Genauigkeit benötigt, um nach dem Finden einer Nullstelle diese von der Funktion zu entfernen.

\subsection{Fazit}
Aufgrund der zuvor genannten Schwächen und Stärken der einzelnen Methoden wähle ich die Weierstraß-Iteration. Dabei hat die Implementierbarkeit eine größere Rolle als die Geschwindigkeit gespielt, da die Geschwindigkeit nur bei sehr komplexen Polynomen wichtig ist und diese meist nicht vorkommen.

\section{Weierstraß-Iteration}
\subsection{Funktionsweise}
\subsubsection{Bedingungen}
Für die Weierstraß-Iteration muss ein normiertes univariates Polynom der Form $p(x) = x^n + a_{n-1} x^{n-1} + \dots + a_1 x + a_0$ gegeben sein. Der Grad des Polynoms $n$ muss größer als oder gleich zwei sein.\footnote{
    vgl. Wikipedia: Weierstraß-(Durand-Kerner)-Verfahren, URL: \url{https://de.wikipedia.org/wiki/Weierstraß-(Durand-Kerner)-Verfahren} (Zuletzt aufgerufen: 29.10.2023)
    \label{ftn:Wikipedia-Weierstraß-Methode}
}

\subsubsection{Gleichungen für die Weierstraß-Iteration}
Bei der Weierstraß-Iteration wird mit jeder Iteration jede Nullstelle in der Regel\footnote{
    vgl. Abschnitt 3.4.2
} etwas genauer. Dafür werden $n$ Gleichungen für die $n$ Nullstellen\footnote{
    Eine Funktion $n$-ten Grades hat nach dem Fundamentalsatz der Algebra genau $n$ komplexe Nullstellen.
} $z_n,z_{n-1},\dots,z_1$ gebildet. Über diese wird iteriert, bis das Endkriterium erreicht ist. Dabei ist $i$ die Anzahl der Iterationen und wird pro Iteration um 1 inkrementiert.
\[z_n^{(i+1)} = z_n^{(i)} - \frac{p(z_n^{(i)})}{\prod_{j=1;j\neq n}^{n}(z_n^{(i)}-z_j^{(i)})}\]
\[z_{n-1}^{(i+1)} = z_{n-1}^{(i)} - \frac{p(z_{n-1}^{(i)})}{\prod_{j=1;j\neq n-1}^{n}(z_{n-1}^{(i)}-z_j^{(i)})}\]
\vspace{0.25mm}
\[\dots\]
\[z_{1}^{(i+1)} = z_1^{(i)} - \frac{p(z_{1}^{(i)})}{\prod_{j=1;j\neq 1}^{n}(z_{1}^{(i)}-z_j^{(i)})}\]
\subsubsection{Startpunkte}
Zunächst müssen alle Startpunkte $z_n^{(0)},z_{n-1}^{(0)},\dots,z_1^{(0)} \in \mathbb{C}$ gesetzt werden. Diese können beliebig gewählt werden. Es muss jedoch beachtet werden, dass Startpunkte für komplexe Nullstellen einen imaginären Teil ungleich Null haben müssen. Außerdem müssen alle Startpunkte unterschiedlich sein, damit bei der ersten Iteration nicht durch Null dividiert wird. Denn bei der Gleichung
\begin{equation*}
    z_k^{(i+1)} = \frac{p(z_{k}^{(i)})}{\prod_{j=1;j\neq k}^{n}(z_{k}^{(i)}-z_j^{(i)})}
\end{equation*}
wird, wenn $z_{k}^{(i)}$ gleich einem anderen $z^{(i)}$ ist, durch Null dividiert.
Die Startpunkte können allerdings auch die Anzahl der Iterationen beeinflussen. Deswegen macht es Sinn, möglichst nahe an den wahrscheinlichen Nullstellen anzufangen.\\
Eine der gängigsten Methoden ist, die Startpunkte in einem Kreis auf der komplexen Ebene zu verteilen. Dabei ist der Radius $r = \sqrt[n]{|a_0|}$ relativ genau. Jedoch ist es sehr ressourcenintensiv Wurzeln zu berechnen. Deshalb kann auch ein ungenauerer Radius genommen werden: $r = |\frac{na_0}{2a_1}| + |\frac{a_{n-1}}{2n}|$.
Auf dem Kreis werden alle Startpunkte gleichmäßig verteilt, indem der Kreis in $n$ Abschnitte geteilt wird, an dessen Anfängen die Startpunkte liegen. Dafür wird das Inkrement $\theta = \frac{2\pi}{n} \text{ bzw. } \frac{360^\circ}{n}$ benötigt. Zuletzt wird noch eine Verschiebung $c$ durchgeführt, um keine reellen Startpunkte zu erhalten. Diese kann beliebig gewählt werden, solange keine reellen Zahlen nach der Verschiebung unter den Startpunkten sind. Bei der folgenden Implementierung wurde $c = \frac{\pi}{2n}$ gewählt.
Um die Startpunkte zu bestimmen wird 
\[z_{k}^{(0)} = r\cos((k-1)\theta+c)+r\sin((k-1)\theta+c)i \text{ für } k=1;2;\dots;n\]
angewandt.\footnote{
    vgl. Oscar Veliz: Durand-Kerner Method: Minute 5:40 (Veröffentlicht am 29.05.2019), URL: \url{https://www.youtube.com/watch?v=5JcpOj2KtWc} (Zuletzt aufgerufen: 30.10.2023) \label{ftn:OscarVilez,5:40}
}

\subsubsection{Endkriterium}
Um zu bestimmen, ob eine Nullstelle gefunden wurde, muss eine gewünschte Genauigkeit $g$ (z.B. $0,0001$) gewählt werden. Diese wird dann mit dem Betrag der Differenz der letzten beiden Schritte der Weierstraß-Iteration verglichen. Wenn $|z_k^{(i-1)}-z_k^{(i)}| < g$ gilt, dann ist die Nullstelle $z_k$ mit der Genauigkeit $g$ gefunden. Dies wird für alle Nullstellen wiederholt. Falls alle Nullstellen genau genug sind ist die Weierstraß-Iteration abgeschlossen.\\
Es kann jedoch auch vorkommen, dass die Weierstraß-Iteration nicht konvergiert. Darauf wird in dem Abschnitt 3.4.2 noch eingegangen. Daher muss eine weitere Endbedingung gestellt sein: Eine maximale Anzahl an Iterationen. Dabei muss man wissen, mit welchen Polynomen man zutun hat. Sind diese sehr komplex, kann es vorkommen, dass die Weierstraß-Iteration weit über Tausend Iterationen benötigt, um eine vernünftige Genauigkeit zu erreichen. In solchen Fällen muss man das Limit sehr hoch setzen. In meiner Implementierung wurde ein Limit von 1000 gesetzt, um für die meisten Polynome zuverlässig zu funktionieren.

%-----------------------------------------------------------------------------------------------------------
\subsection{Beispiel}
Im Folgenden wird die Weierstraß-Iteration an einer Polynomfunktion vierten Grades beispielhaft durchgeführt. Dabei wird die Genauigkeit im Beispiel auf vier Nachkommastellen begrenzt, was zu Ungenauigkeiten kleiner als $0,0001$ führen kann.
\paragraph{Gleichungen}
Für die Funktion
\begin{equation*}
    p(x) = x^4 + 4x^3 - 2x^2 + 3x - 4 = (x-z_1)(x-z_2)(x-z_3)(x-z_4)
\end{equation*}
sind die Nullstellen $z_1,z_2,z_3,z_4 \in \mathbb{C}$ gesucht. Für jede der Nullstellen wird, wie zuvor beschrieben, eine Gleichung gebildet:
\begin{equation*}
    z_1^{(i+1)} = z_1^{(i)}-\frac{p(z_1^{(i)})}{(z_1^{(i)}-z_2^{(i)})(z_1^{(i)}-z_3^{(i)})(z_1^{(i)}-z_4^{(i)})}
\end{equation*}
\begin{equation*}
    z_2^{(i+1)} = z_2^{(i)}-\frac{p(z_2^{(i)})}{(z_2^{(i)}-z_1^{(i)})(z_2^{(i)}-z_3^{(i)})(z_2^{(i)}-z_4^{(i)})}
\end{equation*}
\begin{equation*}
    z_3^{(i+1)} = z_3^{(i)}-\frac{p(z_3^{(i)})}{(z_3^{(i)}-z_1^{(i)})(z_3^{(i)}-z_2^{(i)})(z_3^{(i)}-z_4^{(i)})}
\end{equation*}
\begin{equation*}
    z_4^{(i+1)} = z_4^{(i)}-\frac{p(z_4^{(i)})}{(z_4^{(i)}-z_1^{(i)})(z_4^{(i)}-z_2^{(i)})(z_4^{(i)}-z_3^{(i)})}
\end{equation*}
\paragraph{Startpunkte}
Zuvor müssen allerdings die Startpunkte $z_1^{(0)},z_2^{(0)},z_3^{(0)},z_4^{(0)}$ bestimmt werden. Dabei ist der Radius des Kreises $r = |\frac{na_0}{2a_1}| + |\frac{a_{n-1}}{2na_n}| = \frac{19}{6}$, der Abstand zwischen den Startpunkten $\theta = \frac{2\pi}{n} = \frac{1}{2}\pi$ und die Verschiebung $c = \frac{\pi}{2n} = \frac{1}{8}\pi$. Mit diesen Werten können die Startpunkte berechnet werden. \\
$z_1^{(0)} = r \cos(c) + r \sin(c)i = 2,9256 + 1,2118i$ \\
$z_2^{(0)} = r \cos(\theta+c) + r \sin(\theta+c)i = -1,2118 + 2,9256i$ \\
$z_3^{(0)} = r \cos(2\theta+c) + r \sin(2\theta+c)i = -2,9256 - 1,2118i$ \\
$z_4^{(0)} = r \cos(3\theta+c) + r \sin(3\theta+c)i = 1,2118 - 2,9256i$

\paragraph{Ausführung}
\begin{center}
\begin{tabular}{c|c c c c}
    Iteration & $z_1$ & $z_3$ & $z_2$ & $z_4$ \\
    \hline
    0 & 2,9256 + 1,2118i & -1,2118 + 2,9256i & -2,9256 - 1,2118i & 1,2118 - 2,9256i \\
    1 & 1,2993 + 0,8722i & -1,8873 + 2,0075i & -3,405 - 0,7665i & -0,0069 - 2,1133i \\
    2 & 1,0806 + 0,5889i & -1,1784 + 0,4767i & -4,1287 + 0,3433i & 0,2265 - 1,4089 \\
    3 & 0,7884 + 0,3817i & -0,2138 + 0,7292i & -4,6806 - 0,0989 & 0,1061 - 1,012i \\
    4 & 0,6759 - 0,234i & 0,0243 + 0,9804i & -4,613 - 0,0021i & -0,0872 - 0,9549i \\
    5 & 0,8621 + 0,0197i & -0,1294 + 0,9945i & -4,6149 & -0,1177 - 1,0141i \\
    6 & 0,842 + 0,0004i & -0,1134 + 1,0078i & -4,6149 & -0,1136 - 1,0082i \\
    7 & 0,8421 & -0,1136 + 1,0081i & -4,6149 & -0,1136 - 1,0081i \\
    8 & 0,8421 & -0,1136 + 1,0081i & -4,6149 & -0,1136 - 1,0081i \\
\end{tabular}
\end{center}
\paragraph{Endkriterium}
Zwischen Iteration Sieben und Acht ist die Differenz zwischen den beiden Werten für jede Nullstelle kleiner als die Genauigkeit $0,0001$. Damit ist das Endkriterium erreicht und die Weierstraß-Iteration abgeschlossen. 
Die Nullstellen der Polynomfunktion $p(x) = x^4 + 4x^3 - 2x^2 + 3x - 4$ sind bei ca. $0,8421; -4,6149; -0,1136 - 1,0081i; -0,1136 + 1,0081$. 
\paragraph{Probe}
Zur Überprüfung kann jeder der Werte in $p(x)$ eingesetzt werden.
\begin{displaymath}
    f(0,8421) \approx 0; p(-4,6149) \approx 0; f(-0,1136 - 1,0081i) \approx 0; p(-0,1136 + 1,0081i) \approx 0
\end{displaymath}
Daraus kann geschlossen werden, dass alle vier Werte Annäherungen der Nullstellen der Polynomfunktion $p(x)$ sind. \\
Weiterhin, kann aus dem Fundamentalsatz der Algebra hergeleitet werden, dass alle Nullstellen von $p(x)$ gefunden wurden, da $n$ gleich der Anzahl der gefunden Nullstellen ist. \\
$\Rightarrow$ Die Weierstraß-Iteration war für $p(x)$ erfolgreich.

%--------------------------------------------------------------
\subsection{Herleitung}
\subsubsection{Methode}
Auf die Herleitung der Methode wird, der Einfachheit halber, nur mit reellen Zahlen eingegangen. Jedoch kann die Herleitung, mit wenigen Veränderungen, auch auf komplexe Zahlen übertragen werden.
\paragraph{Visualisierung}
Für die Herleitung der Methode macht es Sinn, den Graphen einer einzelnen Gleichung mit immer genaueren restlichen Nullstellen zu visualisieren. Dabei ist $z_k^{(i)} = x$ und $z_k^{(i+1)} = y$:
\begin{equation}
    g(x) = y = x - \frac{p(x)}{\prod_{j=1;j\neq k}^{n} (x-z_j^{(i)})}
\end{equation}\\
\begin{figure}[H]
    \centering
    \includegraphics[scale=.6]{./BeispielHerleitung.png}
\end{figure}
%-------------------------------------------------------------------------------------------------------------------
Dabei sieht man, dass es sich, wenn die restlichen Nullstellen nicht genau sind, um eine gebrochen rationale Funktion handelt. Das kann auch an der Gleichung gesehen werden, da diese, mit ein wenig umstellen, aus einem Polynom geteilt durch ein anderes Polynom besteht:
\begin{align*}
    g(x) = y &= x - \frac{p(x)}{\prod_{j=1;j\neq k}^{n} (x-z_j^{(i)})} \\
    &= \frac{x}{1} - \frac{\prod_{j=1}^{n} (x-z_j^{(i)})}{\prod_{j=1;j\neq k}^{n} (x-z_j^{(i)})} \\
    &= \frac{x \cdot \prod_{j=1;j\neq k}^{n} (x-z_j^{(i)})}{\prod_{j=1;j\neq k}^{n} (x-z_j^{(i)})} - \frac{\prod_{j=1}^{n} (x-z_j^{(i)})}{\prod_{j=1;j\neq k}^{n} (x-z_j^{(i)})} \\
    &= \frac{x \cdot \prod_{j=1;j\neq k}^{n} (x-z_j^{(i)}) - \prod_{j=1}^{n} (x-z_j^{(i)})}{\prod_{j=1;j\neq k}^{n} (x-z_j^{(i)})} \\
\end{align*}
Wenn die restlichen Nullstellen gefunden wurden handelt es sich um eine waagrechte lineare Funktion. Das liegt daran, dass alle linearen Faktoren des Polynoms, außer dem der gesuchten Nullstelle, gefunden wurden und nun von dem Polynom dividiert werden. Dabei bleibt $x-(x+z_k^{(i)})$ über. Die beiden $x$ fallen weg und es bleibt der konstante Wert $z_k^{(i)}$ über. Daraus folgt, dass die Funktion waagrecht zu der $x$-Achse ist.
%-------------------------------------------------------------------------------------------------------------------
\paragraph{Bedeutung für die Weierstraß-Iteration}
Das Annähern an eine Nullstelle mit jeder Iteration kann man sich mit zwei miteinander verknüpften Annäherungen vorstellen. Einerseits das Annähern an die Nullstelle mit der jeweiligen Gleichung für die Nullstelle und andererseits das Annäheren der Funktion g(x) an die Waagrechte $y=z_k$, mit immer Genaueren restlichen Nullstellen.

\paragraph{Annäherung an die Waagrechte $y=z_k$}
Wie in der Abbildung zuvor zu sehen nährt sich die Funktion mit genaueren Nullstellen an die Funktion $y=z_k$ an. Das hat zur Folge, dass die einzelnen Schritte der Weierstraß-Iteration schneller genau werden. Dabei dient die Annährung an $y=z_k$, da die Gleichung, statt der Nullstelle, genauer wird, mehr als Beschleunigung der Methode, während das Annähren an die Nullstelle mithilfe der Gleichung die eigendliche Annährung bringt.

\paragraph{Annähern an die Nullstelle mithilfe der Gleichung}
Für jedes $z_k^{(i)}$, welches weit genug von den Definitionslücken entfernt ist, gilt, dass $g(z_k^{(i)})$ näher an $z_k$ als $z_k^{(i)}$ ist.\\\\
%-------------------------------------------------------------------------------------------------------------------
Für die Herleitung diese Aussage muss zuerst gezeigt werden, dass $g(x)$ eine waagrechte Asymptote besitzt. Dafür kann man sich die zuvor hergeleitete Gleichung
\begin{equation*}
    g(x) = y = \frac{x \cdot \prod_{j=1;j\neq k}^{n} (x-z_j^{(i)}) - \prod_{j=1}^{n} (x-z_j^{(i)})}{\prod_{j=1;j\neq k}^{n} (x-z_j^{(i)})}
\end{equation*}
angucken. Da die Polynome im Zähler normiert und vom gleichen Grad sind, besitzen beide den Term $a_nx^n$ mit $a_n = 1$. Werden diese Polynome nun miteinander subtrahiert, fällt dieser $n$-te-Term weg, da er bei beiden Polynomen gleich ist. Somit ist der Grad des Differenzpolynoms $n-1$. \\
\begin{equation*}
    g(x) = y = \frac{\text{Polynom vom Grad }n-1}{\text{Polynom vom Grad }n-1}
\end{equation*}
Da der Nenner und Zähler den gleichen Grad haben hat $g(x)$ eine waagrechte Asymptote\footnote{
    vgl. studimup.de: Asymptoten berechnen und erkennen, URL: \url{https://www.studimup.de/abitur/analysis/asymptoten/#waagerecht} (Zuletzt aufgerufen: 04.11.2023) \label{ftn:studimup.de}
}.\\\\
%-------------------------------------------------------------------------------------------------------------------
Wegen dieser waagrechten Asymptote ist $|g(x)|<|x|$ für jedes $x \in \mathbb{R}$ mit ausreichendem Abstand zu den Definitionslücken wahr. Außerdem ist $|g(x)|>|g(z_k)| \land (g(x) \text{ hat das gleiche Vorzeichen wie } z_k)$ für jedes $x \in \mathbb{R}$ mit den gleichen Bedingungen wie zuvor wahr.
Das kann man an der Lage der waagrechten Asymptote sehen. Diese kann man ausrechnen, indem man die Faktoren vor der höchsten Potenz im Zähler durch den Faktor der höchsten Potenz im Nenner teilt\footref{ftn:studimup.de}. Da das Nennerpolynom normiert ist, ist die waagrechte Asymptote gleich des Faktors vor der höchsten Potenz im Zähler. Der Betrag dieser ist größer als $|z_k|$ und hat das gleiche Vorzeichen wie $z_k$.
Daraus folgt, dass jedes $z_k^{(i)}$, welches weit genug von den Definitionslücken entfernt ist, in $g(x)$ eingesetzt einen Wert zurückgibt, welcher näher an $z_k$ als $z_k^{(i)}$ ist.

\paragraph{Schluss}
Das bedeutet für die Weierstraß-Iteration, dass in den meisten Fällen mit jeder Iteration $z_k^{(i+1)}$ näher an $z_k$ als $z_k^{(i)}$ ist. Außerdem wird in den meisten Fällen mit jeder Iteration $g(x)$ genauer. Da an für die Nullstelle $z_k$ die Gleichung $g(z_k) = z_k$ gilt und mit einem höheren $i$ der Wert $z_k^{(i)}$ meist genauer wird, konvergiert die Weierstraß-Iteration meistens. In dem Abschnitt 3.4.2 wird mehr auf die Fälle eingegangen, bei denen die Weierstraß-Iteration nicht konvergiert.

\subsubsection{Wählen der Startpunkte}
Alle Nullstellen einer Polynomfunktion befinden sich auf der abgeschlossen Kreisscheibe  $B(0,M) := \{x \in \mathbb{C} \bigr| |x| \le M\}$ mit $M := 1 + \max\{\frac{|a_j|}{|a_n|}\bigr|\text{\space}j=0,\dots,n-1\}$.\footnote{vgl. \glqq\textit{Sei $P(z) = a_nz^n+a_{n-1}z^{n-1}+\dots+a_1z+a_0$ ein komplexes Polynom vom Grad $n \ge 1$. Dann liegen alle Nullstellen von $P(z)$ in der abgeschlossenen Kreisscheibe $B(0,M) := \{z \in \mathbb{C} \bigr| |z| \le M\}$, wobei $M := 1 + \max\{\frac{|a_j|}{|a_n|}\bigr|\text{\space}j=0,\dots,n-1\}$.}\grqq\space Humboldt-Universität zu Berlin: Helga Baum: Nullstellen komplexer Polynome (6.3. Erster Satz von Cauchy), URL: \url{https://didaktik.mathematik.hu-berlin.de/user/fehlingerl/Helga.pdf} (Zuletzt aufgerufen: 30.10.2023)}  Um möglichst genaue Startpunkte zu bekommen kann man diese auf einer Kreislinie innerhalb der Kreisscheibe anordnen. \\
Dabei gibt es keinen \glqq besten\grqq\space Standard für das Wählen des Radius. Jeder Radius ist für manche Funktionen näher und andere weiter von den Nullstellen entfernt. Je nach den Anforderungen kann man einen eigenen Radius für die Kreislinie nehmen. In meiner Implementierung habe ich mich deshalb für $r = |\frac{na_0}{2a_1}| + |\frac{a_{n-1}}{2n}|$ entschieden.\footref{ftn:OscarVilez,5:40}
Dieser Radius ist relativ ressourcensparend und einfach zu implementieren.

%------------------------------------------------------------------------------------
\subsection{Probleme und Einschränkungen}
\subsubsection{Normiertes Polynom}
Für die Weierstraß-Iteration muss die gegebene Polynomfunktion $p(x)$ in der Form $\prod_{i=1}^n (x-z_i)$ dargestellt werden können. Da $\prod_{i=1}^n (x-z_i)$ ausmultipliziert, in jedem Fall ein normiertes Polynom ergibt, muss $p(x)$ für die Weierstraß-Iteration ein normiertes Polynom sein.
Ist dies nicht der Fall, wird es unmöglich, $p(x)$ in der Form $\prod_{i=1}^n (x-z_i)$ darzustellen. Daraus folgt, dass man die Methode ohne Umwege nicht ausführen kann. \\
Allerdings gibt es eine einfache Lösung für dieses Problem. Wenn man eine Funktion gleich Null setzt, kann man beide Seiten mit einer beliebigen Zahl multiplizieren. Dabei verändert sich das Ergebnis nicht, da $0m = 0$ \space $\forall m \in \mathbb{C}$ ist. Daraus folgt, dass sich die Nullstellen eines Polynoms, wenn dieses mit einer beliebigen Zahl multipliziert wird, nicht verändern. \\
Somit kann man das Polynom, dessen Nullstellen man finden möchte, durch $a_n$ teilen und bekommt ein normiertes Polynom mit den gleichen Nullstellen. Mit diesem kann man dann die Weierstraß-Iteration durchführen.

\subsubsection{Keine generelle Konvergenz}
Die Weierstraß-Iteration ist nicht generell konvergent. Das bedeutet, dass für bestimmte Startpunkte mancher Polynomfunktionen die Methode bei Iterationen gegen Unendlich keinen festen Wert anstrebt. Dabei verfängt sich die Weierstraß-Iteration in periodischen Zyklen\footnote{d. h. die Weierstraß-Iteration wiederholt sich immer wieder in dem gleichen Abstand.}. Es ist nicht generell bestimmt, bei welchen Startpunkten und Polynomen die Weierstraß-Iteration nicht konvergiert. Beispielsweise hat das Polynom $x^3+x+177$ eine offene Menge an Startpunkten, bei denen die Weierstraß-Iteration nicht konvergiert.\footref{ftn:Wikipedia-Weierstraß-Methode} Da dies allerdings, besonders wenn man vernünftige Startpunkte wählt, sehr selten ist, kann es meist vernachlässigt werden. 



\section{Implementierung}
Im Folgenden wird auf die Implementierung der Weierstraß-Iteration in Java eingegangen. Dabei wurden die Klassen \glqq Complex\grqq, \glqq Poly\grqq\space und \glqq Weierstrass\grqq\space implementiert, um komplexe Zahlen, Polynomfunktionen und den Hauptteil der Weierstraß-Iteration darzustellen.
%TODO ausführlicher
%TODO warum Java?

\subsection{Komplexe Zahlen (Complex)}
In Java sind komplexe Zahlen in keinem Grunddatentyp oder erweiterten Datentyp enthalten. Deshalb habe ich mich dazu entschieden, eine eigene Klasse \glqq Complex\grqq\space zu schreiben.

\subsubsection*{Attribute}
Die Attribute \glqq real\grqq\space und \glqq imaginary\grqq\space repräsentieren den reellen und imaginären Teil der komplexen Zahl.

\subsubsection*{Konstruktor}
Der Konstruktor initialisiert eine komplexe Zahl mit einem gegebenen Real- und Imaginärteil.

\subsubsection*{Methoden}
\textit{public double abs()} berechnet und gibt den Betrag der komplexen Zahl zurück.\\
\textit{public Complex round(int accuracyDecimalPlaces)} rundet den Real- und Imaginärteil auf die angegebene Anzahl von Dezimalstellen und gibt eine neue komplexe Zahl zurück.\\
\textit{@Override public String toString()} gibt eine lesbare Repräsentation der komplexen Zahl als String zurück.

\subsubsection*{Statische Methoden}
Die statischen Methoden führen die grundlegenden Rechenoperationen Addition, Subtraktion, Multiplikation, Division und Potenzierung mit komplexen Zahlen durch.

%-----------------------------------------------------------------------------------------
\subsection{Polynomfunktion (Poly)}
Mit der Klasse \glqq Poly\grqq\space werden Polynomfunktionen dargestellt. Die Methoden dieser Klasse %TODO kein Plan was da abgeht

\subsubsection*{Attribute}
In dem Array \glqq coefficients\grqq\space von Gleitkommazahlen (double) sind die Koeffizienten des Polynoms enthalten. Der Index repräsentiert den Exponenten des zugehörigen Terms.
Der Grad des Polynoms wird als Integer in dem Attribut \glqq degree\grqq\space gespeichert. 

\subsubsection*{Konstruktor}
Der Konstruktor \glqq Poly(double... coefficients)\grqq, erstellt ein Polynom mit den gegebenen Koeffizienten.

\subsubsection*{Methoden}
\textit{public void normalise()} normiert das Polynom. Hierbei wird jeder Koeffizient durch den führenden Koeffizienten $a_n$ des Polynoms geteilt. Die Nullstellen des Polynoms ändern sich nicht, aber die Darstellung des Polynoms ändert sich.\\
\textit{public Complex solve(Complex val)} berechnet den Funktionswert des Polynoms für einen gegebenen komplexen Wert \glqq val\grqq. Dabei wird das einfache Horner-Schema\footnote{
    k-achilles.de: Das HORNER-Schema, URL: \url{https://www.k-achilles.de/algorithmen/HORNER-Schema.pdf} (Zuletzt aufgerufen: 18.11.2023)
} verwendet, um die Auswertung effizient durchzuführen. \\
\textit{@Override public String toString()} überschreibt die \glqq toString()\grqq -Methode, um eine lesbare Darstellung des Polynoms auszugeben. Die Methode berücksichtigt verschiedene Fälle, einschließlich der Anzeige von Termen mit positiven oder negativen Koeffizienten und der Auslassung von Termen mit Koeffizienten gleich null.

%----------------------------------------------------------------------------------------------------------
\subsection{Weierstraß-Iteration (Weierstrass)}
%TODO kurze beschreibzung

\subsubsection*{calc(Poly p, Complex[] roots, int index)}
Die \glqq calc-Methode\grqq\space berechnet den nächsten Wert für eine Nullstelle des Polynoms unter Verwendung der Weierstraß-Iterationsformel\footnote{
    vgl. Abschnitt 3.1.2
}. Dabei werden drei Parameter entgegengenommen: das Polynom \glqq p\grqq, für das die Nullstellen berechnet werden, ein Array von komplexen Zahlen \glqq roots\grqq, das die aktuellen Näherungen der Nullstellen enthält und der Index der Nullstelle, die in dieser Iteration aktualisiert wird.\\
Zunächst werden der Zähler (\glqq numerator\grqq) und der Nenner (\glqq denominator\grqq) des Weierstrass-Korrekturterms berechnet. Dann wird dieser Korrekturterm von der aktuellen Nullstelle subtrahiert, um die nächste Näherung zu erhalten.

\subsubsection*{startingPoints(Poly p, int accuracyDecimalPlaces)}
Die \glqq startingPoints\grqq-Methode generiert anhand der Eigenschaften des Polynoms anfängliche Schätzungen für die Nullstellen. Sie nimmt als Parameter das Polynom \glqq p\grqq\space entgegen und gibt ein Array von komplexen Zahlen zurück, das die initialen Schätzungen der Nullstellen enthält.\\
Die Methode verwendet die zuvor beschriebenen Berechnungen, um den Radius und den Winkel für die Platzierung der Anfangsschätzungen auf einem Kreis in der komplexen Ebene zu bestimmen. Die generierten Startpunkte werden als Ausgangspunkte für die Weierstraß-Iteration verwendet.

\subsubsection*{isAccurateEnough(Complex[] newRoots, Complex[] oldRoots, double accuracy)}
Die \glqq isAccurateEnough\grqq-Methode überprüft, ob die Nullstellen ausreichend genau konvergiert sind. Sie vergleicht den Unterschied zwischen den neuen Nullstellen und den alten Nullstellen mit einer vorgegebenen Genauigkeitsgrenze (\glqq accuracy\grqq).\\
Die Methode gibt Wahr zurück, wenn alle Nullstellen innerhalb der Genauigkeitsgrenze konvergiert sind, andernfalls gibt sie Falsch zurück.

\subsubsection*{weierstrass(Poly p)}
Die \glqq Weierstrass\grqq-Methode ist die Hauptmethode der Weierstraß-Iteration. Sie nimmt ein Polynom entgegen und gibt ein Array von komplexen Zahlen mit Nährungen der Nullstellen des Polynoms zurück. Die Methode kann in sechs Schritte unterteilt werden.

\paragraph{a) Polynom normieren:}
Zuerst wird das Polynom normiert, um die Weierstraß-Iteration möglich zu machen. Dafür wird die Methode \glqq normalise()\grqq\space der Polynomfunktion aufgerufen.

\paragraph{b) Genauigkeit und maximale Iterationen festlegen:}
Darauffolgend wird die Genauigkeit mit $10^{-4}$ und maximale Anzahl an Iterationen mit $1000$ festgelegt.

\paragraph{c) Initialisierung der Nullstellen:}
Als nächstes werden mithilfe der zuvor behandelten Funktion \glqq startingPoints(Poly p)\grqq\space die Nullstellen in einem Array Initialisiert.

\paragraph{d) Durchführen der Weierstraß-Iteration:}
Jetzt kann eine Schleife gestartet werden, die solange läuft, bis die Konvergenz erreicht ist oder die maximale Anzahl von Iterationen. In dieser wird die Funktion \glqq calc(Poly p, Complex[] roots, int index)\grqq\space für alle Nullstellen bei jeder Iteration ausgeführt.

\paragraph{e) Probe:}
Nach Abschluss der Iteration wird jede gefundene Nullstelle in das Polynom eingesetzt und ausgegeben, um die Genauigkeit zu überprüfen.

\paragraph{f) Rückgabe der Nährungen der Nullstellen:}
Zuletzt wird das endgültige Array der berechneten Wurzeln zurückgegeben.

\subsection{Input/Output}
Der Input erfolgt mithilfe der Bibliothek \glqq java.util.Scanner\grqq\space in der \glqq main\grqq-Methode. Dabei wird für jeden Term des Polynoms nach dem Koeffizienten gefragt. Es wird bei dem $0$-ten Term begonnen und in aufsteigender Reihenfolge weitergefragt. Wenn der gewünschte Grad erreicht ist, kann der Nutzer \glqq done\grqq\space eingeben und die Weierstraß-Iteration wird mit dem gegebenen Polynom ausgeführt.\\
\\
Darauffolgend wird zuerst die Polynomfunktion zur Kontrolle ausgegeben. Während die Weierstraß-Iteration ausgeführt wird, werden alle Nährungen an die Nullstellen für diese Iteration angegeben. Zuletzt wird die Probe aller Nullstellen ausgegeben. Bei dieser sollte, falls die Weierstraß-Iteration funktiontiert hat, \glqq 0.0\grqq\space für jede Nullstelle herauskommen.\footnote{
    vgl. Abschnitt 6.1
}

\section{Fazit}
%TODO zusammen
\subsection{Zusammenfassung}
%TODO zu kurz
Um die Nullstellen einer Polynomfunktion $n$-ten Grades mithilfe des Computers zu finden kann die Weierstraß-Iteration genutzt werden. Dabei werden $n$ Gleichungen für $n$ Nullstellen gebildet. Diese nehmen die vorherigen Approximationen der Nullstellen und geben meist eine genauere Nullstelle zurück. Sobald die Annährungen die gewünschte Genauigkeit haben, ist die Weierstraß-Iteration beendet. Um zu testen, ob die Methode erfolgreich war, können alle Nährungen in das Polynom eingesetzt werden. In manchen Fällen konvergiert die Weierstraß-Iteration nicht. Aus diesem Grund muss ein Limit an maximalen Iteration angegeben sein.

\subsection{Bezug zu der Problemstellung}
Mit der Implementierung der Weierstraß-Iteration in Java habe ich eine Möglichkeit Nullstellen von Polynomen mithilfe des Computers zu approximieren entwickelt. Es gibt allerdings auch andere Verfahren, die andere Stärken und Schwächen haben. In meiner Implementierung wurde mehr Fokus auf die Lesbarkeit des Codes, statt der Performance gelegt. Letzendlich können jedoch die Nullstellen der meisten Polynomfunktionen mithilfe meiner Implementierung gefunden werden.

\subsection{Reflektion}
%TODO
- Manche dinge nicht vollkommen hergeleitet, weil diese einzelnen Dinge eine eigene Facharbeit bräuchten\\
- Der richtige Beweis ist einfach zu komplex für ein Unterthema in einer Facharbeit.


\section{Anhang}
\subsection{Input/Output Beispiel}
\begin{lstlisting}[keywordstyle=\ttfamily]
PS C:\Development\Facharbeit\src> java Weierstrass
Please input a coefficient for the 0th Term or 'done':
3
Please input a coefficient for the 1th Term or 'done':
-4
Please input a coefficient for the 2th Term or 'done':
2
Please input a coefficient for the 3th Term or 'done':
7
Please input a coefficient for the 4th Term or 'done':
-3
Please input a coefficient for the 5th Term or 'done':
done

polynom: 1.0x^(4) - 2.3333333333333335x^(3) - 0.6666666666666666x^(2) + 1.3333333333333333x - 1.0

iteration: 0
i = 0: 1.6552841624 + 0.6856411497i
i = 1: -0.6856411497 + 1.6552841624i
i = 2: -1.6552841624 - 0.6856411497i
i = 3: 0.6856411497 - 1.6552841624i

iteration: 1
i = 0: 1.8539472642 + 0.5118992343i
i = 1: 0.1470889953 + 1.0987293667i
i = 2: -0.8341324115 - 0.3650474203i
i = 3: 1.1664294853 - 1.2455811807i

iteration: 2
i = 0: 2.1268684556 - 0.0084935444i
i = 1: 0.0818868737 + 0.5809472615i
i = 2: -0.845802451 - 0.1407098505i
i = 3: 0.9703804549 - 0.4317438666i

iteration: 3
i = 0: 2.5309652599 - 0.0492014993i
i = 1: 0.3267937002 + 0.3464210299i
i = 2: -0.9060995604 + 0.0202659974i
i = 3: 0.3816739337 - 0.317485528i

iteration: 4
i = 0: 2.4517265689 - 0.0019358547i
i = 1: 0.436328963 + 0.5486123919i
i = 2: -0.9656563518 - 0.002268555i
i = 3: 0.4109341533 - 0.5444079821i

iteration: 5
i = 0: 2.4512903802 - 2.09016E-5i
i = 1: 0.4208896496 + 0.5020943496i
i = 2: -0.9576001405 - 2.65727E-5i
i = 3: 0.418753444 - 0.5020468753i

iteration: 6
i = 0: 2.451294556 - 8.1E-9i
i = 1: 0.4197084882 + 0.499965846i
i = 2: -0.9573689688 + 7.97E-8i
i = 3: 0.419699258 - 0.4999659176i

iteration: 7
i = 0: 2.4512945632
i = 1: 0.4197037322 + 0.4999622586i
i = 2: -0.9573686941
i = 3: 0.4197037321 - 0.4999622586i

iteration: 8
i = 0: 2.4512945632
i = 1: 0.4197037321 + 0.4999622586i
i = 2: -0.9573686941
i = 3: 0.4197037321 - 0.4999622586i

---- 1.0x^(4) - 2.3333333333333335x^(3) - 0.6666666666666666x^(2) + 1.3333333333333333x - 1.0 ----
Probe: f(2.4512945632) = 0.0
Probe: f(0.4197037321 + 0.4999622586i) = 0.0
Probe: f(-0.9573686941) = 0.0
Probe: f(0.4197037321 - 0.4999622586i) = 0.0
\end{lstlisting}



\subsection{Programmcode}
\subsubsection{Complex.java}
\lstinputlisting[]{../src/Complex.java}
\subsubsection{Poly.java}
\lstinputlisting[]{../src/Poly.java}
\subsubsection{Weierstrass.java}
\lstinputlisting[]{../src/Weierstrass.java}


\subsection{Literaturverzeichnis}
Prof. Dr. Bernd Engelmann: Numerische Mathematik, 2.4 Das Newton-Verfahren und seine Abkömmlinge (S.24)\\
\\
mathe-online.at: Zwischenwertsatz und Bisektionsverfahren, URL: \url{https://www.mathe-online.at/nml/materialien/innsbruck/bisektion/Bisektion.pdf} (Zuletzt aufgerufen: 15.11.2023)\\
\\
Wikipedia: Durand-Kerner method, URL: \url{https://en.wikipedia.org/wiki/Durand-Kerner_method} (Zuletzt aufgerufen: 29.10.2023)\\
\\
Wikipedia: Aberth method, URL: \url{https://en.wikipedia.org/wiki/Aberth_method} (Zuletzt aufgerufen: 18.11.2023)\\
\\
Wikipedia: Weierstraß-(Durand-Kerner)-Verfahren, URL: \url{https://de.wikipedia.org/wiki/Weierstraß-(Durand-Kerner)-Verfahren} (Zuletzt aufgerufen: 29.10.2023)\\
\\
Oscar Veliz: Durand-Kerner Method: Minute 5:40 (Veröffentlicht am 29.05.2019), URL: \url{https://www.youtube.com/watch?v=5JcpOj2KtWc} (Zuletzt aufgerufen: 30.10.2023)\\
\\
studimup.de: Asymptoten berechnen und erkennen, URL: \url{https://www.studimup.de/abitur/analysis/asymptoten/#waagerecht} (Zuletzt aufgerufen: 04.11.2023)\\
\\
Humboldt-Universität zu Berlin: Helga Baum: Nullstellen komplexer Polynome (6.3. Erster Satz von Cauchy), URL: \url{https://didaktik.mathematik.hu-berlin.de/user/fehlingerl/Helga.pdf} (Zuletzt aufgerufen: 30.10.2023)\\
\\
k-achilles.de: Das HORNER-Schema, URL: \url{https://www.k-achilles.de/algorithmen/HORNER-Schema.pdf} (Zuletzt aufgerufen: 18.11.2023)

\end{document}