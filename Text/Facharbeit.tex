\documentclass[12pt]{article}

\usepackage{german}
\usepackage{geometry}
\usepackage[onehalfspacing]{setspace}

\geometry{
    left=3cm,
    right=3cm,
    top=2.5cm,
    bottom=2.5cm
}

\title{Numerisches finden von Nullstellen - Implementierung in Java}
\author{Jakob Rzeppa}

\begin{document}
\begin{titlepage}
	\centering
    {\huge IGSFF LOGO TODO\par}
	{Grünewaldstraße 12a - 38104 Braunschweig - T. 470 5850\par}
	\vspace{1cm}
	{\underline{Seminarfach:}\par}
	\vspace{1cm}
    {\large Facharbeit des Schülers \textit{Jakob Rzeppa} \par mit dem Thema: \par}
    \vspace{1.5cm}
	{\huge Numerisches finden von Nullstellen - Implementierung in Java\par}
	\vspace{2cm}
\end{titlepage}

\tableofcontents

\section{Einleitung}
    \subsection{Problemstellung}

\section{Newton-Horner-Methode}
    \subsection{Herleitung und Funktionsweise}
    \subsubsection{Verfahren}
    \subsubsection{Endbedingungen}
    \subsubsection{Startpunkte}
    \subsection{Beweis}
    \subsection{Probleme}
    \subsection{Implementierung}

\section{Durand-Kerner-Methode} 
    Die Weierstraß-(Durand-Kerner)-Methode ist ein iteratives Verfahren, zur numerischen Bestimmung aller Nullstellen einer Polynomfunktion. Sie wurde zwischen 1859 und 1891 von Karl Weierstrass, als Teil seines Beweises zum Fundamentalsatz der Algebra, entwickelt. E. Durand und Immo Kerner überführten dieses Verfahren in einen Computeralgorythums.

    \subsection{Herleitung und Funktionsweise}
        Die Durand-Kerner-Methode kann Nullstellen von univariaten Polynomfunktionen der Form $f(x) = x^n + a_{n-1} x^{n-1} + ... + a_1 x + a_0$ mit Grad $n > 2$ und Koeffizient $a_n = 1$ numerisch approximieren. Dabei werden sich die Eigenschaften einer Polynomfunktion zugunste gemacht.
    \subsubsection{Difinition}
        Sei $f(x) = x^n + a_{n-1} x^{n-1} + ... + a_1 x + a_0$ ein univariates Polynom mit komplexen Koeffizienten und führenden Koeffizienten $a_n = 1$. Nach dem Fundermentalsatz der Algebra hat $f(x)$ genau $n$ Nullstellen $\xi_n;\xi_{n-1};...;\xi_1 \in C$ hat und in Linearfaktoren zerlegt werden kann. Daraus folgt $f(x) = (x - \xi_n)(x - \xi_{n-1})...(x - \xi_1)$. Nun kann für $\xi_n$ umgestellt werden:
        \begin{displaymath}
            (x - \xi_n)(x - \xi_{n-1})...(x - \xi_1) = f(x)
        \end{displaymath}
        \begin{displaymath}
            \frac{(x - \xi_n)(x - \xi_{n-1})...(x - \xi_1)}{(x - \xi_{n-1})...(x - \xi_1)} = \frac{f(x)}{(x - \xi_{n-1})...(x - \xi_1)}
        \end{displaymath}
        \begin{displaymath}
            x - \xi_n = \frac{f(x)}{(x - \xi_{n-1})...(x - \xi_1)}
        \end{displaymath}
        \begin{displaymath}
            - \xi_n = - x + \frac{f(x)}{(x - \xi_{n-1})...(x - \xi_1)}
        \end{displaymath}
        \begin{displaymath}
            \xi_n = x - \frac{f(x)}{(x - \xi_{n-1})...(x - \xi_1)}
        \end{displaymath}
        Um $\xi_n$ zu bestimmen müssen zunächst aber Startpunkte für alle $\xi$ definiert werden. Dabei können beliebige Komplexe Zahlen gewählt werden, die 


    \subsection{Beispiel}
        In dem nächsten Abschnitt wird die Durand-Kerner-Methode auf eine Polynomfunktion vierten Grades angewand. \\
        Sei $f(x) = x^4 + 4x^3 - 2x^2 + 3x - 4 = (x-r)(x-s)(x-t)(x-u)$ und die Nullstellen $r;s;t;u \in \mathbb{C}$ sind gesucht. \\
        Jetzt kann in die zuvor gezeigte Formel, für die Durand-Kerner-Methode, eingesetzt werden:
        \begin{displaymath}
            r_{n+1} = r_n-\frac{f(r_n)}{(r_n-s_n)(r_n-t_n)(r_n-u_n)}
        \end{displaymath}
        \begin{displaymath}
            s_{n+1} = r_n-\frac{f(s_n)}{(s_n-r_n)(s_n-t_n)(s_n-u_n)}
        \end{displaymath}
        \begin{displaymath}
            t_{n+1} = r_n-\frac{f(t_n)}{(t_n-r_n)(t_n-s_n)(t_n-u_n)}
        \end{displaymath}
        \begin{displaymath}
            u_{n+1} = r_n-\frac{f(u_n)}{(u_n-r_n)(u_n-s_n)(u_n-t_n)}
        \end{displaymath}
        \vspace{0.2cm}
        Nun müssen wir Startpunkte $r_0;s_0;t_0;u_0$ bestimmen. Dabei ist der Radius des Kreises $r$ auf der complexen Ebene $|\frac{na_0}{2a_1}| + |\frac{a_{n-1}}{2na_n}| = |\frac{4 * (-4)}{2 * 3}| + |\frac{4}{2 * 4 * 1}|$

        
        \begin{center}
        Die Genauigkeit liegt bei vier Nachkommastellen \\ und es kann deswegen zu kleinen Abweichungen kommen.
        \begin{tabular}{c|c c c c}
            Iteration & r & s & t & u \\
            \hline
            0 & 2,9256 + 1,2118i & -1,2118 + 2,9256i & -2,9256 - 1,2118i & 1,2118 - 2,9256i \\
            1 & 1,2993 + 0,8722i & -1,8873 + 2,0075i & -3,405 - 0,7665i & -0,0069 - 2,1133i \\
            2 & 1,0806 + 0,5889i & -1,1784 + 0,4767i & -4,1287 + 0,3433i & 0,2265 - 1,4089 \\
            3 & 0,7884 + 0,3817i & -0,2138 + 0,7292i & -4,6806 - 0,0989 & 0,1061 - 1,012i \\
            4 & 0,6759 - 0,234i & 0,0243 + 0,9804i & -4,613 - 0,0021i & -0,0872 - 0,9549i \\
            5 & 0,8621 + 0,0197i & -0,1294 + 0,9945i & -4,6149 & -0,1177 - 1,0141i \\
            6 & 0,842 + 0,0004i & -0,1134 + 1,0078i & -4,6149 & -0,1136 - 1,0082i \\
            7 & 0,8421 & -0,1136 + 1,0081i & -4,6149 & -0,1136 - 1,0081 \\
            8 & 0,8421 & -0,1136 + 1,0081i & -4,6149 & -0,1136 - 1,0081 \\
        \end{tabular}
        \end{center}
        Da sich zwischen Iteration acht und neun die Werte (Genauigkeit = 4 Nachkommastellen) gleich blieben ist die Durand-Kerner-Methode abgeschlossen. Die Nullstellen der Polynomfunktion $f(x) = x^4 + 4x^3 - 2x^2 + 3x - 4$ sind bei ca. $0,8421; -4,6149; -0,1136 - 1,0081i; -0,1136 + 1,0081$. Um das zu überpfüfen kann jeder der Werte in $f(x)$ eingesetzt werden. Wenn die Ergebnisse Null oder Zahlen sehr nah an Null sind dann sind die Nullstellen gefunden.
        \begin{displaymath}
            f(0,8421) \approx 0; f(-4,6149) \approx 0; f(-0,1136 - 1,0081i) \approx 0; f(-0,1136 + 1,0081i) \approx 0
        \end{displaymath}
        Daraus kann geschlossen werden, dass alle vier Werte Approximationen der Nullstellen der Polynomfunktion $f(x)$ sind. Außerdem, kann aus dem Fundermentalsatz der Algebra, welcher besagt, dass eine univariate Polynomfunktion genau $n$ (Grad der Funktion) Nullstellen hat, geschlossen werden, dass alle Nullstellen von $f(x)$ gefunden wurden, da $n$ gleich der Anzahl der gefunden Nullstellen ist.
        
    \subsubsection{Endbedingungen}
    \subsubsection{Startpunkte}
    \subsection{Beweis}
    \subsection{Probleme}
    \subsection{Implementierung}

\section{Vergleich und Anwendung}

\section{Fazit}

\section{Code}

\end{document}