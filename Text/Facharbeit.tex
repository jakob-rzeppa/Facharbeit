\documentclass[12pt]{article}

\usepackage{german}
\usepackage{geometry}
\usepackage[onehalfspacing]{setspace}

\geometry{
    left=3cm,
    right=3cm,
    top=2.5cm,
    bottom=2.5cm
}

\usepackage{listings}
\usepackage{xcolor}

\lstset{
    tabsize = 4, %% set tab space width
    showstringspaces = false, %% prevent space marking in strings, string is defined as the text that is generally printed directly to the console
    numbers = left, %% display line numbers on the left
    commentstyle = \color{green}, %% set comment color
    keywordstyle = \color{blue}, %% set keyword color
    stringstyle = \color{red}, %% set string color
    rulecolor = \color{black}, %% set frame color to avoid being affected by text color
    basicstyle = \small \ttfamily , %% set listing font and size
    breaklines = true, %% enable line breaking
    numberstyle = \tiny,
}

\title{Numerisches finden von Nullstellen - Implementierung in Java}
\author{Jakob Rzeppa}

\begin{document}
\begin{titlepage}
	\centering
    {\huge IGSFF LOGO TODO\par}
	{Grünewaldstraße 12a - 38104 Braunschweig - T. 470 5850\par}
	\vspace{1cm}
	{\underline{Seminarfach:}\par}
	\vspace{1cm}
    {\large Facharbeit des Schülers \textit{Jakob Rzeppa} \par mit dem Thema: \par}
    \vspace{1.5cm}
	{\huge Numerisches finden von Nullstellen - Implementierung in Java\par}
	\vspace{2cm}
\end{titlepage}

\tableofcontents

\section{Einleitung}
    \subsection{Problemstellung}

\section{Newton-Horner-Methode}
    \subsection{Herleitung und Funktionsweise}
    \subsubsection{Verfahren}
    \subsubsection{Endbedingungen}
    \subsubsection{Startpunkte}
    \subsection{Beweis}
    \subsection{Probleme}
    \subsection{Implementierung}

\section{Durand-Kerner-Methode} 
    Die Weierstraß-(Durand-Kerner)-Methode ist ein iteratives Verfahren, zur numerischen Bestimmung aller Nullstellen einer Polynomfunktion. Sie wurde zwischen 1859 und 1891 von Karl Weierstrass, als Teil seines Beweises zum Fundamentalsatz der Algebra, entwickelt. E. Durand und Immo Kerner überführten dieses Verfahren in einen Computeralgorythums.

    \subsection{Herleitung und Funktionsweise}
        Die Durand-Kerner-Methode kann Nullstellen von univariaten Polynomfunktionen der Form $f(x) = x^n + a_{n-1} x^{n-1} + ... + a_1 x + a_0$ mit Grad $n > 2$ und Koeffizient $a_n = 1$ numerisch approximieren. Dabei werden sich die Eigenschaften einer Polynomfunktion zugunste gemacht.
    \subsubsection{Definition}
        Sei $f(x) = x^n + a_{n-1} x^{n-1} + ... + a_1 x + a_0$ ein univariates Polynom mit komplexen Koeffizienten und führenden Koeffizienten $a_n = 1$. Nach dem Fundermentalsatz der Algebra hat $f(x)$ genau $n$ Nullstellen $\xi_n;\xi_{n-1};...;\xi_1 \in C$ hat und in Linearfaktoren zerlegt werden kann. Daraus folgt $f(x) = (x - \xi_n)(x - \xi_{n-1})...(x - \xi_1)$. Nun kann für $\xi_n$ umgestellt werden:
        \begin{displaymath}
            (x - \xi_n)(x - \xi_{n-1})...(x - \xi_1) = f(x)
        \end{displaymath}
        \begin{displaymath}
            \frac{(x - \xi_n)(x - \xi_{n-1})...(x - \xi_1)}{(x - \xi_{n-1})...(x - \xi_1)} = \frac{f(x)}{(x - \xi_{n-1})...(x - \xi_1)}
        \end{displaymath}
        \begin{displaymath}
            x - \xi_n = \frac{f(x)}{(x - \xi_{n-1})...(x - \xi_1)}
        \end{displaymath}
        \begin{displaymath}
            - \xi_n = - x + \frac{f(x)}{(x - \xi_{n-1})...(x - \xi_1)}
        \end{displaymath}
        \begin{displaymath}
            \xi_n = x - \frac{f(x)}{(x - \xi_{n-1})...(x - \xi_1)}
        \end{displaymath}
        Um $\xi_n$ zu bestimmen müssen zunächst aber Startpunkte für alle $\xi$ definiert werden. Dabei können beliebige Komplexe Zahlen gewählt werden, die 


    \subsection{Beispiel}
        Im nächsten Abschnitt wird die Durand-Kerner-Methode an einer Polynomfunktion vierten Grades beispielhaft durchgeführt. Dabei wird die Genauigkeit im Beispiel auf vier Nachkommastellen begrenzt, was zu Ungenauigkeiten kleiner als $0,0001$ führen kann. \\
        Für die Funktion $f(x) = x^4 + 4x^3 - 2x^2 + 3x - 4 = (x-s)(x-t)(x-u)(x-v)$ sind die Nullstellen $r;s;t;u \in C$ gesucht. Auf jede der Nullstellen wird die zuvor beschriebene Formel angewendet:
        \begin{displaymath}
            s_{n+1} = s_n-\frac{f(s_n)}{(s_n-t_n)(s_n-u_n)(s_n-v_n)}
        \end{displaymath}
        \begin{displaymath}
            t_{n+1} = t_n-\frac{f(t_n)}{(t_n-s_n)(t_n-u_n)(t_n-v_n)}
        \end{displaymath}
        \begin{displaymath}
            u_{n+1} = u_n-\frac{f(u_n)}{(u_n-s_n)(u_n-t_n)(u_n-v_n)}
        \end{displaymath}
        \begin{displaymath}
            v_{n+1} = v_n-\frac{f(v_n)}{(v_n-s_n)(v_n-t_n)(v_n-u_n)}
        \end{displaymath}
        Nun müssen die Startpunkte $s_0;t_0;u_0;v_0$ bestimmt werden. Dabei ist der Radius des Kreises $r = |\frac{na_0}{2a_1}| + |\frac{a_{n-1}}{2na_n}| = \frac{19}{6}$, der Abstand zwischen den Startpunkten $\theta = \frac{2\pi}{n} = \frac{1}{2}\pi$ und die Verschiebung $c = \frac{\pi}{2n} = \frac{1}{8}\pi$. Mit diesen Werten können die Startpunkte berechnet werden. \\
        $s_0 = r \cos(c) + r \sin(c)i = 2,9256 + 1,2118i$ \\
        $t_0 = r \cos(\theta+c) + r \sin(\theta+c)i = -1,2118 + 2,9256i$ \\
        $u_0 = r \cos(2\theta+c) + r \sin(2\theta+c)i = -2,9256 - 1,2118i$ \\
        $v_0 = r \cos(3\theta+c) + r \sin(3\theta+c)i = 1,2118 - 2,9256i$ \\\\
        Daraus ergibt sich:
        \begin{center}
        \begin{tabular}{c|c c c c}
            Iteration & r & s & t & u \\
            \hline
            0 & 2,9256 + 1,2118i & -1,2118 + 2,9256i & -2,9256 - 1,2118i & 1,2118 - 2,9256i \\
            1 & 1,2993 + 0,8722i & -1,8873 + 2,0075i & -3,405 - 0,7665i & -0,0069 - 2,1133i \\
            2 & 1,0806 + 0,5889i & -1,1784 + 0,4767i & -4,1287 + 0,3433i & 0,2265 - 1,4089 \\
            3 & 0,7884 + 0,3817i & -0,2138 + 0,7292i & -4,6806 - 0,0989 & 0,1061 - 1,012i \\
            4 & 0,6759 - 0,234i & 0,0243 + 0,9804i & -4,613 - 0,0021i & -0,0872 - 0,9549i \\
            5 & 0,8621 + 0,0197i & -0,1294 + 0,9945i & -4,6149 & -0,1177 - 1,0141i \\
            6 & 0,842 + 0,0004i & -0,1134 + 1,0078i & -4,6149 & -0,1136 - 1,0082i \\
            7 & 0,8421 & -0,1136 + 1,0081i & -4,6149 & -0,1136 - 1,0081 \\
            8 & 0,8421 & -0,1136 + 1,0081i & -4,6149 & -0,1136 - 1,0081 \\
        \end{tabular}
        \end{center}
        Zwischen Iteration acht und neun bleiben die Werte gleich. Damit ist das Endkriterium erreicht und damit ist die Durand-Kerner-Meethod abgeschlossen. Die Nullstellen der Polynomfunktion $f(x) = x^4 + 4x^3 - 2x^2 + 3x - 4$ sind bei ca. $0,8421; -4,6149; -0,1136 - 1,0081i; -0,1136 + 1,0081$. \\
        Zur Überpfüfung kann jeder der Werte in $f(x)$ eingesetzt werden.
        \begin{displaymath}
            f(0,8421) \approx 0; f(-4,6149) \approx 0; f(-0,1136 - 1,0081i) \approx 0; f(-0,1136 + 1,0081i) \approx 0
        \end{displaymath}
        Daraus kann geschlossen werden, dass alle vier Werte Annährungen der Nullstellen der Polynomfunktion $f(x)$ sind. \\
        Aus dem Fundermentalsatz der Algebra kann geschlossen werden, dass alle Nullstellen von $f(x)$ gefunden wurden, da $n$ gleich der Anzahl der gefunden Nullstellen ist.
    \subsubsection{Endbedingungen}
    \subsubsection{Startpunkte}
    \subsection{Beweis}
    \subsection{Probleme}
    \subsection{Implementierung}
    \begin{lstlisting}[language=Java]
private static Complex denominator(Complex[] roots, int index) {
    List<Complex> factors = new ArrayList<Complex>();
    for (int i = 0; i < roots.length; i++) {
        if (i == index) {
            continue;
        }
        factors.add(Complex.minus(roots[index], roots[i]));
    }
    return Complex.multiply(factors);
}
    \end{lstlisting}
    \subsubsection{Einleitung}
    \subsubsection{Komplexe Zahlen}
    \subsubsection{Polynomfunktionen}
    \subsubsection{Durand-Kerner für eine Nullstellen}
    \subsubsection{Startpunkte}
    \subsubsection{Durnad-Kerner für alle Nullstellen}
    \subsubsection{Endbedingungen}
    \subsubsection{Testen}

\section{Vergleich und Anwendung}

\section{Fazit}

\section{Code}

\end{document}