\documentclass[12pt]{article}

\usepackage{german}
\usepackage{geometry}
\usepackage[onehalfspacing]{setspace}

\geometry{
    left=3cm,
    right=3cm,
    top=2.5cm,
    bottom=2.5cm
}

\title{Numerisches finden von Nullstellen - Implementierung in Java}
\author{Jakob Rzeppa}

\begin{document}
\begin{titlepage}
	\centering
    {\huge IGSFF LOGO TODO\par}
	{Grünewaldstraße 12a - 38104 Braunschweig - T. 470 5850\par}
	\vspace{1cm}
	{\underline{Seminarfach:}\par}
	\vspace{1cm}
    {\large Facharbeit des Schülers \textit{Jakob Rzeppa} \par mit dem Thema: \par}
    \vspace{1.5cm}
	{\huge Numerisches finden von Nullstellen - Implementierung in Java\par}
	\vspace{2cm}
\end{titlepage}

\tableofcontents

\section{Einleitung}
    \subsection{Problemstellung}

\section{Newton-Horner-Methode}
    \subsection{Herleitung und Funktionsweise}
    \subsubsection{Verfahren}
    \subsubsection{Endbedingungen}
    \subsubsection{Startpunkte}
    \subsection{Beweis}
    \subsection{Probleme}
    \subsection{Implementierung}

\section{Durand-Kerner-Methode}
    \subsection{Herleitung und Funktionsweise}
    \subsubsection{Verfahren}
    \subsubsection{Endbedingungen}
    \subsubsection{Startpunkte}
    \subsection{Beweis}
    \subsection{Probleme}
    \subsection{Implementierung}

\section{Vergleich und Anwendung}

\section{Fazit}

\section{Code}

\end{document}